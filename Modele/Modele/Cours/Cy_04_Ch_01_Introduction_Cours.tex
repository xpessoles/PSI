%%%% Paramétrage du cours %%%%
\def\xxactivite{Cours}
\def\xxauteur{\textsl{Xavier Pessoles}}

\fichefalse
\proftrue
\tdfalse
\courstrue

\def\xxnumchapitre{Chapitre 1 \vspace{.2cm}}
\def\xxchapitre{\hspace{.12cm} Introduction à la dynamique du solide indéformable}

\def\xxcompetences{%
\textsl{%
\textbf{Savoirs et compétences :}\\
\begin{itemize}[label=\ding{112},font=\color{ocre}] 
\item \textit{Res1.C2} : principe fondamental de la dynamique;
\item \textit{Res1.C1.SF1} : proposer une démarche permettant la détermination de la loi de mouvement.
\end{itemize}
}}


\def\xxfigures{
\includegraphics[width=4cm]{newton}%images/prot_01
%\\

 \textit{Isaac Newton -- 1643 - 1727.}
}%figues de la page de garde



\input{style/new_pagegarde}

\setlength{\columnseprule}{.1pt}

\vspace{2cm}
\pagestyle{fancy}
\thispagestyle{plain}



\section{Introduction}
\begin{obj} ~\\
\lipsum[1]
\end{obj}
%\subsection{Définitions}





\section[PFD : cas général]{Principe Fondamental de la Dynamique : cas général}

\subsection{Principe Fondamental de la Dynamique}

\begin{definition}[Énoncé du Principe Fondamental de la Dynamique] ~\\
\lipsum[1]
\end{definition}

\begin{rem}
\lipsum[1]
\end{rem}


\subsection{Équations de mouvement}
\begin{exemple}

\lipsum[1]

\end{exemple}

\begin{demo}
\lipsum[1]
\end{demo}


\begin{term}
\lipsum[1]
\end{term}

\begin{py}
\lipsum[1]
\end{py}



\subsection{Théorèmes généraux}



\begin{theorem}[Théorème de la résultante dynamique]
			
\end{theorem}

\begin{rem}
\lipsum[1]
\end{rem}

\begin{warn}
\lipsum[1]
\end{warn}

\begin{methode}
\lipsum[1]
\end{methode}


\begin{savoir}
\lipsum[1]
\end{savoir}


\begin{corrige}
\lipsum[1]
\end{corrige}

\begin{thebibliography}{2}
   \bibitem[1]{ref1} Xavier Pessoles, \url{http://github.com/xpesssoles/}.
\end{thebibliography}

%\end{document}



