\documentclass[10pt,fleqn]{article} % Default font size and left-justified equations
\usepackage[%
    pdftitle={Modélisation systèmes multiphysiques : Modélisation linéaire et non linéaire},
    pdfauthor={Xavier Pessoles}]{hyperref}

\input{style/new_style}
\input{style/macros_SII}

\fichetrue
%\fichefalse

\proftrue
%\proffalse

%\tdtrue
\tdfalse

\courstrue
%\coursfalse



% -------------------------------------
% Déclaration des titres
% -------------------------------------

\def\discipline{Sciences \\Industrielles de \\ l'Ingénieur}
\def\xxtete{Sciences Industrielles de l'Ingénieur}

\def\classe{\textsf{Cy 02}}
\def\xxnumpartie{Cycle 02}
\def\xxpartie{Modéliser les systèmes asservis dans le but de prévoir leur comportement}

\def\xxnumchapitre{Chapitre 1 \vspace{.2cm}}
\def\xxchapitre{\hspace{.12cm} Stabilité des systèmes}

\def\xxposongletx{2}
\def\xxposonglettext{1.45}
\def\xxposonglety{19}%16

\def\xxonglet{Cycle 02}

\def\xxactivite{Cours}
\def\xxauteur{\textsl{Xavier Pessoles}}

\def\xxcompetences{%
\textsl{%
\textbf{Savoirs et compétences :}\\
\begin{itemize}[label=\ding{112},font=\color{ocre}] 
\item \textit{Mod3.C2 : } pôles dominants et réduction de l’ordre du modèle : principe, justification
\item \textit{Res2.C4 : } stabilité des SLCI : définition entrée bornée -- sortie bornée (EB -- SB)	
\item \textit{Res2.C5 : } stabilité des SLCI : équation caractéristique	
\item \textit{Res2.C6 : } stabilité des SLCI : position des pôles dans le plan complexe
\item \textit{Res2.C7 : } stabilité des SLCI : marges de stabilité (de gain et de phase)
\end{itemize}
}}

	
		
	
		


\def\xxfigures{
%\includegraphics[width=1.4\textwidth]{images/matlab}%images/prot_01
%\\
%\textit{Modèle du pilote hydraulique avec pilotage interactif.}
}%figues de la page de garde

\def\xxpied{%
Cycle 02 -- Modéliser les SLCI dans le but de prévoir leur comportement\\
Chapitre 1 -- \xxactivite%
}

\setcounter{secnumdepth}{5}
%---------------------------------------------------------------------------


\begin{document}
%\chapterimage{png/Fond_SLCI}
%\input{style/new_pagegarde}
\setlength{\columnseprule}{.1pt}

\vspace{2cm}
\pagestyle{fancy}
\thispagestyle{plain}

Équations du moteur à courant continu : 
$$u(t) = Ri(t) + e(t)$$
$$e(t) = k_e(t) \omega_m(t)$$
$$J_e =\dfrac{\text{d}\omega_m(t)}{\text{d} t}$$
$$c_m(t)=k_t i(t)$$

\includestandalone{images/Schema_1_entree_F_R}

\includestandalone{images/Schema_1_entree_2F_R}

\includestandalone{images/Schema2entrees}

\includestandalone{images/Schema_1_entree_2F}


\includestandalone{images/Schema2entrees_2F_R}

%%%%%%%%%%%%%%%%%%%%%%%%%%%%%%%%%%%%%%%ù
\footnotesize
\begin{center}
\begin{tikzpicture}
\sbEntree{E}

\sbBloc[3]{b0}{$A_1$}{E}
    \sbRelier[$Q(p)$]{E}{b0}


\sbComp{c1}{b0}
    \sbRelier{b0}{c1}

\sbBloc[1]{b1}{$A_2$}{c1}
    \sbRelier{c1}{b1}
    
\sbBloc[3]{b11}{$A_3$}{b1}
    \sbRelier[$\Sigma(p)$]{b1}{b11}


\sbComph{c2}{b11}
    \sbRelier{b11}{c2}

\sbBloc{b2}{$A_4$}{c2}
    \sbRelier{c2}{b2}
    

\sbSortie[4]{S}{b2}
    \sbRelier{b2}{S}
    \sbNomLien[0.8]{S}{$X(p)$}
  
\sbRenvoi{b2-S}{c1}{}

\draw [latex-] (c2) --++ (0,1) node[left] {$F_R(p)$};

\end{tikzpicture}
\end{center}
\normalsize
%%%%%%%%%%%%%%%%%%%%%%%%%%%%%%%%%%%%%%%%


%%%%%%%%%%%%%%%%%%%%%%%%%%%%%%%%%%%%%%%%%%%
\footnotesize
\begin{center}
\begin{tikzpicture}
\sbEntree{E}

\sbBloc[3]{b1}{$H_1$}{E}
    \sbRelier[$Q(p)$]{E}{b1}


\sbComph{c1}{b1}
    \sbRelier{b1}{c1}
  
\sbBloc{b2}{$H_2$}{c1}
    \sbRelier{c1}{b2}

\sbSortie{S}{b2}
    \sbRelier{b2}{S}
    \sbNomLien[0.8]{S}{$X(p)$}

%\sbBloc[3]{b11}{$A_3$}{b1}
%    \sbRelier[$\Sigma(p)$]{b1}{b11}
%
%
%\sbSumh{c2}{b11}
%    \sbRelier{b11}{c2}
%
%\sbBloc{b2}{$A_4$}{c2}
%    \sbRelier{c2}{b2}
%    
%
%\sbSortie[4]{S}{b2}
%    \sbRelier{b2}{S}
%    \sbNomLien[0.8]{S}{$X(p)$}
%  
%\sbRenvoi{b2-S}{c1}{}
%
\draw [latex-] (c1) --++ (0,1) node[left] {$F_R(p)$};

\end{tikzpicture}
\end{center}
\normalsize
%%%%%%%%%%%%%%%%%%%%%%%%%%%%%%%%%%%%%%%%%%%%%%%%%ù

%%%%%%%%%%%%%%%%%%%%%%%%%%%%

\footnotesize
\begin{center}
\begin{tikzpicture}
\sbEntree{E}

\sbBloc[3]{b0}{$A_1 A_2 A_3$}{E}
    \sbRelier[$Q(p)$]{E}{b0}

\sbComph{c2}{b0}
    \sbRelier{b0}{c2}
    
\sbComp{c1}{c2}
    \sbRelier{c2}{c1}

\sbBloc{b1}{$A_4$}{c1}
    \sbRelier{c1}{b1}
    

\sbSortie[4]{S}{b1}
    \sbRelier{b1}{S}
    \sbNomLien[0.8]{S}{$X(p)$}
      
      
\sbDecaleNoeudy[4]{S}{U}
\sbDecaleNoeudx[-2]{U}{U2}
\sbBlocr{r1}{$A_2 A_3$}{U2}


\sbRelieryx{b1-S}{r1}
\sbRelierxy{r1}{c1}


%    
%\sbBloc[3]{b11}{$A_3$}{b1}
%    \sbRelier[$\Sigma(p)$]{b1}{b11}
%
%
%\sbComph{c2}{b11}
%    \sbRelier{b11}{c2}
%
%\sbBloc{b2}{$A_4$}{c2}
%    \sbRelier{c2}{b2}
%    
%
%\sbRenvoi{b2-S}{c1}{}

\draw [latex-] (c2) --++ (0,1) node[left] {$F_R(p)$};

\end{tikzpicture}
\end{center}
\normalsize
%%%%%%%%%%%%%%%%%%%%%%%%%%%%

%%%%%%%%%%%%%%%%%%%%%%%%%%%%
\footnotesize
\begin{center}
\begin{tikzpicture}[scale=0.75, every node/.style={transform shape}]
\sbEntree{E}

\sbBloc[3]{b0}{$K'_C$}{E}
    \sbRelier[$\Theta_c(p)$]{E}{b0}


\sbComp[5]{c1}{b0}
    \sbRelier[$V_c(p)$]{b0}{c1}

\sbBloc[1]{b1}{$C(p)$}{c1}
    \sbRelier{c1}{b1}
    
\sbBloc[2.5]{b2}{$H_{\text{SV}}$}{b1}
    \sbRelier[$V(p)$]{b1}{b2}

\sbBloc[2.5]{b3}{$H_{1}$}{b2}
    \sbRelier[$Q(p)$]{b2}{b3}
    
\sbComph[3.5]{c2}{b3}
    \sbRelier{b3}{c2}

\sbBloc[1]{b4}{$H_2$}{c2}
    \sbRelier{c2}{b4}
    
\sbBloc[2.5]{b5}{$H_{\text{CIN}}$}{b4}
    \sbRelier[$X(p)$]{b4}{b5}

\sbSortie[2]{S}{b5}
    \sbRelier{b5}{S}
    \sbNomLien[0.8]{S}{$\Theta(p)$}
  

\sbDecaleNoeudy[4]{b3}{U}
\sbBlocr{r1}{$K_C$}{U}

\sbRelieryx{b5-S}{r1}
\sbRelierxy[$V_m(p)$]{r1}{c1}

%\sbRenvoi{b2-S}{c1}{}

\draw [latex-] (c2) --++ (0,1) node[left] {$F_R(p)$};

\end{tikzpicture}
\end{center}

\normalsize

%%%%%%%%%%%%%%%%%%%%%%%%%%%%%%%%%%%%%%%%%%%ùù





\end{document}



